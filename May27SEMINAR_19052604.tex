
\documentclass[dvipdfmx,10pt,notheorems]{beamer}

\usepackage{amsmath,amsfonts,amsthm,amssymb,amscd}
\usepackage[all]{xy}
\def\objectstyle{\displaystyle}

\usepackage{mathrsfs}

%for inserting pictures
\usepackage{graphicx}

\usepackage{atbegshi,amsmath,amssymb}

%for inserting diagrams with loops
\usepackage{mathtools}

%for inserting an image of the Universe on the backside of the slide
\usepackage{wallpaper}

\AtBeginShipoutFirst{\special{pdf:tounicode EUC-UCS2}}
%\AtBeginShipoutFirst{\special{pdf:tounicode 90ms-RKSJ-UCS2}}
\usepackage{minijs}
% \usepackage{otf}
\renewcommand{\kanjifamilydefault}{\gtdefault}
\usetheme{Berkeley}
% to let the row a little small at the first slide
\makeatletter
\beamer@headheight=1.5\baselineskip
\makeatother
\setbeamertemplate{navigation symbols}{}

\def\languagename{ja}
\deftranslation[to=ja]{theorem}{定理}
\deftranslation[to=ja]{algorithm}{アルゴリズム}

\def\proofname{\bf Proof}
\newtheorem{theorem}{Theorem}[section]
\newtheorem{axiom}[theorem]{Axiom}
\newtheorem{attention}[theorem]{Attention}
\newtheorem{counterexample}[theorem]{Counterexample}
\newtheorem{definition}[theorem]{Definition}
\newtheorem{example}[theorem]{Example}
\newtheorem{examples}[theorem]{例}
\newtheorem{proposition}[theorem]{Proposition}
\newtheorem{lemma}[theorem]{Lemma}
\newtheorem{corollary}[theorem]{Corollary}
\newtheorem{problem}[theorem]{問題}
\newtheorem{given}[theorem]{仮定}
\newtheorem{algorithm}[theorem]{Algorithm}
\newtheorem{examplequestion}[theorem]{例題}


\usepackage{graphicx}
\usepackage{amsmath,amssymb}
% for Fonts (extra)
\usepackage{txfonts}

\usepackage{listings}
%\newcommand{\qed}{\hbox{\rule{6pt}{6pt}}}
\newcommand{\mono}{\rightarrowtail}
\newcommand{\epi}{\twoheadrightarrow}
\renewcommand{\#}{^\sharp}
\newcommand{\dom}{\mbox{\rm dom}}
%\newcommand{\mod}{\mbox{\rm mod}}
% \newcommand{\mod}{\mbox{\ mod\ }}
\renewcommand{\Im}{\mbox{\rm Im}}
\newcommand{\Pfn}{\mbox{\bf Pfn}}
%\newcommand{\Rel}{\mbox{\bf Rel}}
%\newcommand{\Set}{\mbox{\bf Set}}
\newcommand{\FSet}{\mbox{\bf FSet}}
\newcommand{\Vect}{\mbox{\bf Vect}}
\newcommand{\Poset}{\mbox{\bf Poset}}
\newcommand{\coeq}{\mbox{\rm coeq}}
\newcommand{\eq}{\mbox{\rm eq}}
\newcommand{\rel}{\rightharpoondown}
\newcommand{\id}{\mbox{id}}
%
\newcommand{\Graph}{\mbox{\bf Graph}}
\newcommand{\G}{\mbox{\bf G}}
\newcommand{\C}{\mbox{\cal C}}
\newcommand{\E}{\mbox{\bf E}}
\newcommand{\M}{\mbox{\bf M}}
\newcommand{\Obj}{\mbox{\cal Obj}}
\newcommand{\Mor}{\mbox{\cal Mor}}
\newcommand{\Dyn}{\mbox{\bf Dyn}}
\newcommand{\Mach}{\mbox{\bf Mach}}
%
\newtheorem{defi}{{\bf 定義}}
\newtheorem{prop}{{\bf 命題}}
\newtheorem{conjecture}{{\bf 予想}}
\newtheorem{thm}{{\bf 定理}}
\newtheorem{exam}{{\bf 例}}
\newtheorem{fac}{{\bf 事実}}
\newtheorem{lem}{{\bf 補題}}
\newtheorem{coro}{{\bf 系}}
\newtheorem{excer}{{\bf 練習}}


\DeclareMathOperator{\CH}{CH}
\DeclareMathOperator{\Clos}{Clos}
\DeclareMathOperator{\ev}{ev}
\DeclareMathOperator{\Ker}{Ker}
\DeclareMathOperator{\Image}{Im}
%\DeclareMathOperator{\End}{End}
\DeclareMathOperator{\Gr}{Gr}
%\DeclareMathOperator{\Tab}{Tab}
%\DeclareMathOperator{\sgn}{sgn}
\DeclareMathOperator{\STab}{STab}
\DeclareMathOperator{\Map}{Map}
\DeclareMathOperator{\Id}{Id}
%\DeclareMathOperator{\id}{id}
%\DeclareMathOperator{\rank}{rank}
\DeclareMathOperator{\Hom}{Hom}
\DeclareMathOperator{\ob}{ob}
\DeclareMathOperator{\Rel}{Rel}
\DeclareMathOperator{\Set}{Set}
\DeclareMathOperator{\Setm}{Setm}
%\DeclareMathOperator{\Top}{Top}
\DeclareMathOperator{\tr}{tr}
\DeclareMathOperator{\triv}{triv}


\def\x{\mathbf{x}}


\title{Categories of topological spacies isomorphic to categories of relational algebras for a monad}
%\subtitle[Role, Origin, Apps, Def, e.t.c...]{Role, Origin, Apps, Def, e.t.c...}
\author[Naoto Agawa]
{Naoto Agawa}
\institute[Kyushu Univ., Math Dept.
]{\url{}\\%溝口研究室
}
\date{Monday, 27 May, 2019}

\begin{document}

\begin{frame}\frametitle{}
  \titlepage
\end{frame}

\begin{frame}\frametitle{}
  \tableofcontents
\end{frame}

%%%%%%%%%%%%%%%%%%%%%%%%%%%%%%%%%%%%%%%
%\section*{目次}
%\begin{frame}\frametitle{概要}
%  \tableofcontents
%\end{frame}

%\section{はじめに}
%\begin{frame}\frametitle{}
%\begin{figure}[h]
%Diaconis Sturmfelsの写真と論文
%\includegraphics[width=50mm]{Diaconis.eps}
%\includegraphics[width=50mm]{sturmfels.eps}
%\includegraphics[width=50mm]{paper.eps}
%\end{figure}
%\begin{center}
%``Algebraic algorithms for sampling from conditional distributions"

%\end{center}
%\end{frame}

\section{Introduction1}
	\begin{frame}{Aim for this seminar}
		\begin{proposition}[$\mathrm{M. Barr},1970$]
				$$ {\rm Top} \cong {\rm Rel(\mathbf{U})} $$
		\end{proposition}
		{\small
			 \begin{itemize}
				\item[$\dagger$]  M. Barr, {\sl Relational algebra}, Lecture Notes in Math., 
				137:39-55, 1970.
			 \end{itemize}
		 }
		\begin{itemize}
			 \item We try to formally prove his result with relational calculus.
		\end{itemize}
	\end{frame}







	\begin{frame}{Tools in this seminar}
		\begin{itemize}
			\item Categories
			\item Functors
			\item Natural transformations
				\begin{itemize}
					\item Vertical composites
					\item "Quasi-" horizontal composites
				\end{itemize}
			\item Adjoint functors
			\item Monads
			\item Relational algebras
			\item Filters
		\end{itemize}
	\end{frame}






	\begin{frame}{What is category theory?}
		\begin{itemize}
			\item Definition of category as one thoery in math
				\begin{itemize}
					\item
				\end{itemize}
		\end{itemize}
	\end{frame}






	\begin{frame}{Another topic for category thoery}
		\begin{itemize}
			\item Beck's theorem
				\begin{itemize}
					\item Required tools
						\begin{itemize}
							\item Fundamental ideas on the previous slide
							\item Universality
							\item The comparison functor
							\item Coequalizers
							\item Coequalizer creators
						\end{itemize}
				\end{itemize}
			\item Implemented FORGETTING types ($\cdots$ a variable absorbs everything)
			\begin{itemize}
			\item categories
				\begin{itemize}
					\item associativity
					\item identity
				\end{itemize}
			\item[$\rightarrow$]
			\item functors
				\begin{itemize}
					\item Law of operators-preservation
				\end{itemize}
			\item natural transformations
			\end{itemize}
		\end{itemize}
	\end{frame}



	\begin{frame}{Aim00 for category thoery}
%
%
\begin{itemize}
%
%
\item Areas of mathematics
%
%
\begin{itemize}
%
\item Set theory
%
\item Linear algebra
%
\item Group theory

				\item Ring theory
				\item Module theory
				\item Topology
				\item Algebraic geometry
			\end{itemize}
		\end{itemize}
	\end{frame}







	\begin{frame}{Origin of category thoery}
		\begin{itemize}
			\item ORIGIN
				\begin{itemize}
					\item Sprout
						\begin{itemize}
							\item (PAPER) S. Eilenberg and S. MacLane, Natural Isomorshisms in
							Group Theory, Proceedings of the National Academy of Sciences,
							28(1942), 537-543.
							\item "Frequently in modern mathematics there occur phenomena of "naturality":
			a "natural" isomorphism between two groups or between
			two complexes, a "natural" homeomorphism of two spaces and the
			like. We here propose a precise definition of the "naturality" of such
			correspondences, as a basis for an appropriate general theory."
							\item[$\rightarrow$] They might want to formulize "naturality" between one mathematical
			flamework and another flamework; i.e. a NATURAL ISOMORPHISM
			between two functors in the current category theory.
							\item Ref: https://qiita.com/snuffkin/items/ecda1af8dca679f1c8ac
						\end{itemize}
					\item Topology (Homology)
						\begin{itemize}
							\item (PAPER) Samuel Eilenberg and Saunders Mac Lane, General theory
							of natural equivalences. Transactions of the American Mathematical
							Society 58 (2) (1945), pp.231-294.
							\item They must find it important to DEVELOP an ALGEBRAIC
							FLAMEWORK focused on the feature of homomorphisms or
							mappings, by CALCULATION of the TOPOLOGICAL INVARIANT
							from a series of GROUP HOMOMORPHISMs.
							\item Ref: Book of Proffesor Y. Kawahara
						\end{itemize}
				\end{itemize}
		\end{itemize}
	\end{frame}






	\begin{frame}{Definition of natural isomorphism}

	\end{frame}






	\begin{frame}{Apps of category theory}
		\begin{itemize}
			\item APPLIED AREAS:
				\begin{itemize}
					\item Quantum topology
						\begin{itemize}
							\item Happy outcomes:
							\item[$\rightarrow$]  Tangles have a great interaction with various algebraic properties for
		their invariants, which allows us to have more deep study for
		substantial properties of links.
							\item[$\rightarrow$]  Helps us to see the quantum invariants as the functors from the
		category of tangles to a category, where a tangle is a subset of links
		(, in intuition, where a link is a collection of multiple knots and a knot
		is one closed string).
							\item[$\rightarrow$] We can generate a invariant for a tangle every time you choose a
		special category (called ribbon category) and its object, where in
		most cases we choose ribbon category with myriads of elements.
							\item "A polynomial invariant for knots via non Neumann algebras",
		Bulletin of American Mathematical Society (N. S.) 12 (1985), no. 1, pp.103-111.
							\item Awarded the fields medal on 1990 at Kyoto with "For the proof of
		Hartshorne's conjecture and his work on the classification of
		three-dimensional algebraic varieties."
							\item cf. At the same meeting a Japanese proffesor Shigefumi Mori was
		awarded with "For the proof of Hartshorne's conjecture and his work
		on the classification of three-dimensional algebraic varieties."
						\end{itemize}
				\end{itemize}
		\end{itemize}
	\end{frame}






	\begin{frame}{Apps of cateory}
		\begin{itemize}
			\item Denotational semantics for programming languages
			\item Group theory
			\item Mathematical physics (especially, quantum physics) based on operator algebras
			\item Galois theory and physics
			\item Logic
			\item Algebraic geometry
			\item Algebraic topology
			\item Representation theory
			\item System biology
		\end{itemize}
	\end{frame}






	\begin{frame}{Features on cateory theory}
			\begin{itemize}
					\item set theory $\cdots$ point-oriented;
							\begin{itemize}
									\item $^\forall x,x'\in s(f), f(x)=f(x')\Rightarrow x=x'$;
									\item $^\forall y\in t(f), ^\exists x\in s(f) s.t. f(x)=y$;
									\item $\emptyset_X$;
								\item $\{a\}\in X$;
							\end{itemize}
					\item category thoery $\cdots$ arrow-oriented;
							\begin{itemize}
									\item $^\forall g_1, g_2: W\rightarrow s(f), f\circ g_1=f\circ g_2 \Rightarrow
									g_1=g_2$ $ \mbox{ ( assuming }W\mbox{ is a set with }W=s(g_1), W=s(g_2))$;
									\item $^\forall g_1, g_2: t(f)\rightarrow Z,  g_1 \circ f = g_2 \circ f \Rightarrow
									g_1=g_2$ $\mbox{ ( assuming }Z\mbox{ is a set with }Z=t(g_1), W=t(g_2))$;
									\item $^\forall X, ^{\exists !} f:X \rightarrow \emptyset_X$;
									\item $^\forall Y, ^{\exists !} f:\{a\}\rightarrow Y$;
							\end{itemize}
			\end{itemize}
	\end{frame}
	
	
	
	\begin{frame}{Features on cateory theory}
			\begin{itemize}
					\item set theory $\cdots$ point-oriented;
							\begin{itemize}
									\item $X\times Y=\{(x,y);x\in X, y\in Y\}$;
							\end{itemize}
					\item category thoery $\cdots$ arrow-oriented;
							\begin{itemize}
									\item For sets $X$ and $Y$, a set $X\times Y$ is called
									the cartesian product if the following condition satisfies:
											\begin{itemize}
													\item There exists arrows
														$
														\xymatrix{
														X
														& X\times Y
														\ar[l]_{\pi_l}
														\ar[r]^{\pi_r}
														& Y
														}
														$
													such that the univarsality
														\begin{equation*}
														^{\exists!}(f,g):Z\rightarrow
														X\times Y,~s.t.~\pi_l(f,g)=f \wedge \pi_r(f,g)=g
														\end{equation*}
													holds for a set $Z$ and arrows
														$
														\xymatrix{
														X
														& Z
														\ar[l]_{f}
														\ar[r]^{g}
														& Y
														}
														$
													.
													\begin{equation*}
														\begin{gathered}
															\xymatrix{
															X
															\ar@{}[drr]|{\rotatebox{0}{$\circlearrowright$}}
															&&
															X\times Y
															\ar[ll]_{\pi_l}
															\ar[rr]^{\pi_r}
															&&
															Y
															\ar@{}[dll]|{\rotatebox{0}{$\circlearrowleft$}}
															\\
															&&
															Z
															\ar@{.>}[u]_{^{\exists!}(f,g)}
															\ar@/^18pt/[ull]^{f}
															\ar@/_18pt/[urr]_{g}
															&&
															}
														\end{gathered}
													\end{equation*}
								\end{itemize}
						\end{itemize}
			\end{itemize}
	\end{frame}



	\begin{frame}{Definition of categories}
			\begin{itemize}
					\item The definition of a category
							\begin{itemize}
									\item A pair $\mathcal{C}=(\mathcal{O},(\mathcal{C}(a,b))_{(a,b)\in\mathcal{O}^2},
									 (\circ_{(a,b,c)})_{(a,b,c)\in\mathcal{O}^3})$
									with the three following concepts
											\begin{itemize}
													\item $\mathcal{O}$:a set;
													\item $(\mathcal{C}(a,b))_{(a,b)\in\mathcal{O}^2}$:
													a family of sets with the index set $\mathcal{O}^2$;
													\item $(\circ_{(a,b,c)})_{(a,b,c)\in\mathcal{O}^3}$:
													a family of maps with the index set $\mathcal{O}^3$;
											\end{itemize}
									is called a category if the following conditions
											\begin{itemize}
													\item $\mathcal{C}(a,b)$ is disjoint i.e. $(a,b)\neq (a',b')\Rightarrow
													\mathcal{C}(a,b)\cap\mathcal{C}(a',b')\neq\emptyset$;
													\item $\circ_{(a,b,c)}$:$\mathcal{C}(a,b)\times\mathcal{C}(b,c)\rightarrow
													\mathcal{C}(c,a)$:
													a map; omitted by for convinience from here onward;
													\item $^\forall a\in\mathcal{O}, ^\exists \id_a\in\mathcal{C}(a,a)~s.t.~
													^\forall b\in\mathcal{O}, ^\forall f\in\mathcal{C}(b,a),
													^\forall g\in\mathcal{C}(a,b), \id_a\circ f=f, g\circ\id_a=g$
													\item  $^\forall a,b,c,d\in\mathcal{O}, ^\forall f\in\mathcal{C}(a,b),
													 ^\forall g \in \mathcal{C}(b,c), ^\forall h \in\mathcal{C}(c,d),
													(h\circ g)\circ f=h\circ (g\circ f)$;
											\end{itemize}
							\end{itemize}
					all satisfy $\mathcal{C}$.
			\end{itemize}
	\end{frame}



\section{Introduction2}
	\begin{frame}{GO THROUGH THIS SLIDE}
			\begin{definition}[Quivers]					
					A pair $Q=(\mathcal{O},\mathcal{M},s,t)$ is called a {\bf quiver} or an {\bf oriented graph} if
					following conditions
							\begin{itemize}
									\item $\mathcal{O}$ and $\mathcal{M}$ are sets;
									\item $s:\mathcal{M}\rightarrow\mathcal{O}$ and $t:\mathcal{M}\rightarrow\mathcal{O}$
									are maps;
							\end{itemize}
					are satisfied.
					An element of $\mathcal{O}$ is called a {\bf vertex}, and that of  $\mathcal{M}$ an {\bf arrow}.
					For an arrow $f\in\mathcal{M}$, $s(f)$ is called a {\bf source} of $f$ and $t(f)$ is called a {\bf target} of $f$.
					$Q$ is called a {\bf finite quiver} if $\mathcal{O}$ and $\mathcal{M}$ are finite.
			\end{definition}
			\begin{figure}
					\includegraphics[width=4cm]{2000px-Directed_graph.png}
					\caption{an oriented graph}
					\label{ImageOfAnOrientedGraph}
			\end{figure}
	\end{frame}
	
	
	
	\begin{frame}{Definition of categories}
			\begin{definition}[Quivers]
					Let
							\begin{itemize}
									\item $\mathcal{O}$ and $\mathcal{M}$ be sets.
									\item $s:\mathcal{M}\rightarrow\mathcal{O}$ and $t:\mathcal{M}\rightarrow\mathcal{O}$
									be maps.
							\end{itemize}
					Then, a quadruplet $Q=(\mathcal{O},\mathcal{M},s,t)$ is called a {\bf quiver} or an {\bf oriented graph}, where
					an element of
							\begin{itemize}
									\item  $\mathcal{O}$ is called a {\bf vertex};
									\item  $\mathcal{M}$ is called an {\bf arrow};
							\end{itemize}
					and the image
							\begin{itemize}
									\item $s(f)$ is called a {\bf source} of $f$
									\item $t(f)$ is called a {\bf target} of $f$
							\end{itemize}
					for an arrow $f\in\mathcal{M}$.\\
					$Q$ is called a {\bf finite quiver} if $\mathcal{O}$ and $\mathcal{M}$ are finite.
			\end{definition}
	\end{frame}



	\begin{frame}{Definition of categories}
			\begin{figure}
					\includegraphics[width=4cm]{2000px-Directed_graph.png}
					\caption{an oriented graph}
					\label{ImageOfAnOrientedGraph}
			\end{figure}
	\end{frame}
	
		
		
	\begin{frame}{Definition of categories}			
			\begin{definition}[Paths in a quiver]
					Let a quadruplet $Q=(\mathcal{O},\mathcal{M},s,t)$ be a quiver,
					and, for $n\geq 2$, $\mathcal{M}_n(Q)$ a set defined by
							$$
									\mathcal{M}_n(Q):=\{
									(f_1,\cdots,f_n)\in\mathcal{M}^n; s(f_i)=t(f_{i+1}), 1\leq i\leq n-1
									\},
							$$
					where $\mathcal{M}_0(Q):=\mathcal{O}$ and $\mathcal{M}_1(Q):=\mathcal{M}$.\\
					Then, an element of $\mathcal{M}_n(Q)$ is called a {\bf path} of length $n$ in $Q$, and is described as follows:
							\begin{equation*}
									\begin{gathered}
															\xymatrix{
															v_n
															\ar[r]^{f_n}
															&
															v_{n-1}
															\ar[r]^{f_{n-1}}
															&
															v_{n-2}
															\ar[r]^{f_{n-2}}
															&
															\cdots
															\ar[r]^{f_2}
															&
															v_{1}
															\ar[r]^{f_1}
															&
															v_0
															}.
									\end{gathered}
							\end{equation*}
					Moreover,
							\begin{itemize}
									\item a path $(f_1,\cdots,f_n)\in\mathcal{M}^n(Q)$ is denoted by $f_1 \cdots f_n$.
									\item $\mathcal{M}_n(Q)$ is denoted by $\mathcal{M}_n$.	
							\end{itemize}
			\end{definition}
	\end{frame}		
	
	
	
	\begin{frame}{Definition of categories}
			\begin{definition}[Categories]
							Let
									\begin{itemize}
											\item $(\mathcal{O},\mathcal{M},s,t)$ be a quiver,
											where
													\begin{itemize}
															\item it is denoted by $\mathrm{Quiver}(\mathcal{C})$
															\item
															$\mathcal{M}_n(\mathrm{Quiver}(\mathcal{C}))$
															is denoted
															by $\mathcal{M}_n$ or $\mathcal{M}_n(\mathcal{C})$
													\end{itemize}											
											\item $\circ:\mathcal{M}_2\rightarrow\mathcal{M}$ be a map, where
											$\circ(f,g)$ is denoted by $f\circ g$.
									\end{itemize}
							 Then, a quintette $\mathcal{C}=(\mathcal{O},\mathcal{M},s,t,\circ)$ is called a {\bf category} if
									\begin{itemize}
										\item $s(f\circ g)=s(g)$ and $t(f\circ g)=t(f)$ hold for a path $(f,g)\in\mathcal{M}_2$
										\item (associativity) $(f\circ g)\circ h=f\circ (g\circ h)$ holds for a path $(f,g,h)\in\mathcal{M}$
										\item (identity) there exists a map $1:\mathcal{O}\rightarrow\mathcal{M}$
										both $f\circ 1_{s(f)}=f$ and $1_{t(f)}\circ f=f$ hold, where $1(A)$ is denoted by
										$1_A$ for a vertex $A\in\mathcal{O}$
									\end{itemize}
					all satisfy $\mathcal{C}$.
					\begin{equation*}
									\begin{gathered}
															\xymatrix{
															v_3
															\ar[r]^{h}
															&
															v_2
															\ar[r]^{g}
															&
															v_{1}
															\ar[r]^{f}
															&
															v_0
															}.
									\end{gathered}
					\end{equation*}
			\end{definition}
	\end{frame}
	
	
	
	\begin{frame}{Definition of categories}
			\begin{definition}[from the previous slide (other helpful words)]
					\begin{itemize}
							\item An element in $\mathcal{O}$ is called an {\bf object}.
							\item $\mathcal{O}$ is called {\bf a set of objects}
							in $\mathcal{C}$ and denoted by $\mathrm{Ob}(\mathcal{C})$.
							\item An element in $\mathcal{M}$ is called a {\bf morphism} or an {\bf arrow}.
							\item $\mathcal{M}$ is called {\bf a set of morphisms} in $\mathcal{C}$
							 and denoted by $\mathrm{Mor}(\mathcal{C})$.
							\item For a morphism $f\in\mathcal{M}$,
							 		\begin{itemize}
											\item $s(f)$ is called a {\bf source} of $f$, or a {\bf domain} of $f$.
											\item $t(f)$ is called a {\bf target} of $f$, or a {\bf codomain} of $f$.
									\end{itemize}
							\begin{equation*}
									\begin{gathered}
															\xymatrix{
															source
															\ar[r]^{f}
															&
															target
															}
									\end{gathered}
							\end{equation*}
					\end{itemize}
			\end{definition}
	\end{frame}
	
	
	
	\begin{frame}{Definition of categories}
			\begin{definition}[from the previous slide (other helpful words)]
					\begin{itemize}
							\item $\circ$ is called a {\bf composition}.
							\item  $f\circ g$ is called a {\bf composite} of $f$ and $g$.
							\item 
							$A\in \mathrm{Ob}(\mathcal{C})$ is denoted by $A \in \mathcal{C}$ for simplicity
							as long as there is no risk of confusion.
							\item	For sets $A$ and $B$ in $\mathcal{C}$,
									\begin{itemize}
											\item a subset $s^{-1}(A) \cap t^{-1}(B)$ of $\mathrm{Mor}(\mathcal{C})$
											is called a {\bf set of
											morphisms} from $A$ to $B$, and it
											is denoted by $\mathrm{Hom}_{\mathcal{C}}(A,B)$.
											\item an element in $\mathrm{Hom}_{\mathcal{C}}(A,B)$
											is called a {\bf morphism} from $A$ to $B$.
											\item $f\in\Hom_{\mathcal{C}}(A,B)$ is denoted by $f:A\rightarrow B$.
									\end{itemize}
					\end{itemize}
			\end{definition}
	\end{frame}
	
	
	
	\begin{frame}{Definition of categories}
			\begin{lemma}
					Let
							\begin{itemize}
									\item $\mathcal{C}:=(\mathcal{O}, \mathcal{M}, s,t,\circ)$ be a category,
							\end{itemize}
					and let
							\begin{itemize}
									\item $U_A:=\Hom_{\mathcal{C}}(A,A)$ be a set
									\item $G_A:=(U_A, \circ)$ the diad
							\end{itemize}
					 for an object $A\in\mathcal{O}$.\\
					Then,
					$G_A$ is a monoid with the identity element $1_A$.
			\end{lemma}
	\end{frame}
	
	
	\begin{frame}
			\begin{proof}
					We have that
					\begin{itemize}
						\item ${}^\forall f,g,h\in U_A, (h\circ g)\circ f = h\circ (g\circ f)~(\because
						\mbox{the associativity on }\mathcal{C})$.
						\item putting on $\id_{G_A}:=1_A$, then
								$$i \circ \id_{G_A} = f \circ 1_A = f $$
						and
								$$ \id_{G_A}\circ f = 1_A \circ f = f$$
						hold for an arrow $ f\in U_A$
						$(\because
						\mbox{the identity on }\mathcal{C})$.
					\end{itemize}
			\end{proof}
	\end{frame}
	
	
		
	\begin{frame}
			\begin{corollary}
					On the previous lemma, the arrow $1_A$ and
					the map $1:\mathcal{O}\rightarrow \mathcal{M}$ are respectively {\bf unique},
					because of the uniqueness of the identity in a monoid.
			\end{corollary}			
			\begin{definition}
					On the previous corollary, $1_A$ is called the {\bf identity} or the {\bf identity morphism} on $A$.
			\end{definition}
	\end{frame}



		
	\begin{frame}{Definition of categories}
			\begin{definition}[Categories (traditional)]
					Let 
							\begin{itemize}
									\item $\mathcal{O}:\mbox{a set}$.
									\item $(\mathcal{C}(a,b))_{(a,b)\in\mathcal{O}^2}:
									\mbox{a family of sets with the index set }\mathcal{O}^2$.
									\item $(\circ_{(a,b,c)})_{(a,b,c)\in\mathcal{O}^3}:
									\mbox{a family of maps with the index set }\mathcal{O}^3$.
							\end{itemize}
					Then,
					the triad $\mathcal{C}=(\mathcal{O},(\mathcal{C}(a,b))_{(a,b)\in\mathcal{O}^2},
					(\circ_{(a,b,c)})_{(a,b,c)\in\mathcal{O}^3})$
					is called a {\bf category} if
							\begin{itemize}
									\item $\{\mathcal{C}(a,b)\}_{(a,b)\in \mathcal{O}^2}$
									is disjoint i.e. $(a,b)\neq (a',b')\Rightarrow
									\mathcal{C}(a,b)\cap\mathcal{C}(a',b')\neq\emptyset$
									\item $\circ_{(a,b,c)}:\mathcal{C}(a,b)\times\mathcal{C}(b,c)\rightarrow
									\mathcal{C}(c,a)$ is
									a map,
									and it is omitted by $\circ$ for convinience from here onward
									\item (identity)
									$^\forall a\in\mathcal{O}, ^\exists \id_a\in\mathcal{C}(a,a)~s.t.~
									^\forall b\in\mathcal{O}, ^\forall f\in\mathcal{C}(b,a),
									^\forall g\in\mathcal{C}(a,b), \id_a\circ f=f, g\circ\id_a=g$
									\item  (associativity)
									$^\forall a,b,c,d\in\mathcal{O}, ^\forall f\in\mathcal{C}(a,b),
									 ^\forall g \in \mathcal{C}(b,c), ^\forall h \in\mathcal{C}(c,d),
									(h\circ g)\circ f=h\circ (g\circ f)$
							\end{itemize}
					all satisfy $\mathcal{C}$.
			\end{definition}
	\end{frame}



	\begin{frame}{Definition of categories}
			\begin{lemma}[Cantor]
					Let $A$ be a set. Then $|A|<|\mathcal{P}(A)|$ holds, where $\mathcal{P}(A)$ is the power set of $A$.
			\end{lemma}
			\begin{proof}
					\begin{itemize}
							\item If $A=\emptyset$, then we have $\mathcal{P}(A)=\{\emptyset\}$,
							which yields $|A|<|\mathcal{P}(A)|$.
							\item If $A\neq \emptyset$, we only have to take an injection but  is a bijection.
							Let $f:A\rightarrow \mathcal{P}(A)$ be a map defined by
									$$
											f(x)=\{x\},
									$$
							then this map is an injection, which yields $|A|\leq|\mathcal{P}(A)|$.
							Thus, we only have to verify that $f$ is not a bijection.\\
							(GO TO THE NEXT SLIDE.)
					\end{itemize}
			\end{proof}			
	\end{frame}
	
	
	
	\begin{frame}{Definition of categories}
			\begin{proof}
					(FROM THE PREVIOUS SLIDE)\\
					Assuming $|A|=|\mathcal{P}(A)|$ holds
					in order to use the proof of contradiction,
					then we have a bijection $g:A\rightarrow\mathcal{P}(A)$
					by definition.
					By the way, $g(a)$ is a subset of $A$ because $g(a)\in\mathcal{P}(A)$ for all $a\in A$.
					Thus, we have\\
							\begin{center}
									$a\in g(a) \vee a\notin g(a)$,
							\end{center}
					so let $R$ be a set defined by
							$$
									R=\{x\in A; x\notin g(x)\},
							$$
					then we have $R\in\mathcal{P}(A)$.
					Note that $g$ is a surjection, we find $\alpha\in A$ satisfied with $g(\alpha)=R$.
					\begin{itemize}
							\item
							If $\alpha\in R$, we have $\alpha\notin g(\alpha)=R$ by the definition of $R$.
							\item
							Conversely, given $\alpha \notin R$, we have $\alpha\in g(\alpha)=R$.
					\end{itemize}
					By this contradiction, we get $|A|\neq |\mathcal{P}(A)|$,
					which presents the desired equation $|A|\lneq|\mathcal{P}(A)|$.
			\end{proof}
	\end{frame}
	
	
	
	\begin{frame}{Definition of categories}
			\begin{proposition}
					Let $(A_i)_{i\in I}$ be a family of sets with the index set $I$.\\
					Then there exists a set not isomorphic to any of the set $A_j$ for an index $j\in I$.
			\end{proposition}
			\begin{proof}
					By the previous lemma,
					the proof completes when you take the power set of $A_j$.
			\end{proof}
			\begin{itemize}
					\item[$\rightarrow$] A collection of all sets is too large to be a set, and is neither a category.
					\item (In this seminar, we do not use a category such that $\mathcal{O}$ and $\mathcal{M}$ are
					too large to be sets.)
					\item A conept "universe" was yielded.
					\item An universe is a set among which we can consider any operations.
			\end{itemize}
	\end{frame}
	
	
	
	\begin{frame}{Definition of categories}
			\begin{figure}
					\includegraphics[width=1cm]{19051202.jpeg}
					\label{ImageOfTheUniverse}
			\end{figure}		
			\begin{axiom}[Universes]
					There exists an universe $\mathcal{U}$ such that $X\in\mathcal{U}$ holds for a set $X$,
					where
					a  {\bf (Grothendieck) universe} is a set $\mathcal{U}$  with the following properties:
							\begin{itemize}
									\item $\mathbb{N}=\{0,1,2,\cdots\}\in\mathcal{U}$.
									\item $^\forall x,y, x\in y, y\in\mathcal{U} \Rightarrow x\in\mathcal{U}$.
									\item $I\in\mathcal{U}, f:I\rightarrow \mathcal{U}:$a map
									$\Rightarrow \bigcup_{i\in I}f(i)\in\mathcal{U}$.
									\item $x\in\mathcal{U} \Rightarrow$ $\mathcal{P}(x)\in\mathcal{U}$,
									where $\mathcal{P}(x)$ is the power set of $x$.
							\end{itemize}
			\end{axiom}
	\end{frame}



	\begin{frame}{Examples of categories}
			\begin{example}
					By the defininition of quivers, we can take $\mathcal{C}$, an unique pair of sets defined by
							$$
									\mathrm{Ob}(\mathcal{C})=\emptyset
							$$
							$$
									\mathrm{Mor}(\mathcal{C})=\emptyset,
							$$
					and this is called the {\bf empty category}. \\
			\end{example}
			\begin{example}
					There exists a category with single object and single arrow (the identity), and is denoted by
					$\mathbf{1}$.
							$$
									\xymatrix{\cdot \ar@(dr,ur)_{\id}}
							$$
			\end{example}
	\end{frame}




	\begin{frame}{Examples of categories}
			\begin{example}
					There exists a category with two objects $a,b$ and just one arrow									
					not the identity, and is denoted by
					$\mathbf{2}$.
							$$
														\xymatrix{
														a
														\ar@(dl,ul)^{\id}
														\ar[r]^{}
														&
														b
														\ar@(dr,ur)_{\id}
														}
							$$
			\end{example}
			\begin{example}
					There exists a category with three objects, non-identity arrows of which are arranged as the following traiangle,
					and it is denoted by
					$\mathbf{3}$.
					$$
														\xymatrix{
														&
														\cdot \cdot
														\ar[rd]^{}
														\ar@(ul,ur)^{\id}
														&
														\\
														\cdot
														\ar[ru]^{}
														\ar[rr]^{}
														\ar@(dl,ul)^{\id}
														&
														&
														\cdot \cdot \cdot
														\ar@(dr,ur)_{\id}
														}
					$$
			\end{example}
	\end{frame}



	\begin{frame}{Examples of categories}
			\begin{example}
					Let $X$ be a set, then $\mathcal{C}$, a pair of sets, defined by
							$$\mathrm{Ob}(\mathcal{C})=X$$
							$$\mathrm{Mor}(\mathcal{C})=\{1_x;x\in X\}$$
					is a category, and it is called a {\bf descrete category}.
					In fact,
							$$\mathcal{C}(x,x)=\{1_x\}$$
							$$\mathcal{C}(x,y)=\emptyset~(x\neq y)$$
					hold for an element $x\in X$.
							$$
														\xymatrix{
														\cdots \cdots
														&
														x
														\ar@(ul,ur)^{\id}
														&
														y
														\ar@(ul,ur)^{\id}
														&
														z
														\ar@(ul,ur)^{\id}
														&
														\cdots \cdots
														}
							$$
			\end{example}
	\end{frame}




	\begin{frame}
			\begin{example}[a category with single object]
							Given a monoid $M$, then $\mathcal{C}$, a pair of sets,
							is to be a category if defined as follows:
							$$
									\mathrm{Ob}(\mathcal{C}):=(\mbox{the underlying set of }M)
							$$
							$$
									\mathrm{Mor}(\mathcal{C}):=\left\{*\overset{\id_{*}}{\rightarrow} *\right\}
							$$
							$$
									\circ_{\mathcal{C}}:=(\mbox{the operator in }\mathrm{M})~
									(\circ_\mathcal{C}\mbox{ is a map on category }\mathcal{C}).
							$$
							Thus, we can construct a category with single object.
			\end{example}
	\end{frame}


	
	\begin{frame}{Examples of categories}
			\begin{proposition}[Categories with single object]
					Let $\mathbf{M}$ with single object be a subcategory of $\mathrm{Cat}$.
					Then, we have $\mathrm{Mon}\cong\mathbf{M}$.
					$$
														\xymatrix{
														\ast
														\ar@(dr,ur)_{}
														}
							$$

			\end{proposition}
			$\rightarrow$ I'm going to strictly show this fact later, because we have to have more concepts in category theory e.g.
			\begin{itemize}
					\item functors
					\item natural transformations
					\item the isomorphic-density for a functor
					\item other more concepts...
			\end{itemize}
	\end{frame}
	
	
	
	\begin{frame}{GO THROUGH THIS SLIDE}
			\begin{definition}
					The denotation $\circ_{\mathcal{X}}$ is composition in a category $\mathcal{X}$.
			\end{definition}
			\begin{definition}
					{\bf subcategory}
			\end{definition}
			\begin{definition}
					{\bf full subcategory}
			\end{definition}
	\end{frame}
	
	
	
	\begin{frame}{GO THROUGH THIS SLIDE}
			\begin{definition}
					{\bf functor}
			\end{definition}
			\begin{definition}
					the {\bf composite}
			\end{definition}
			\begin{definition}
					the {\bf identity functor}
			\end{definition}
			\begin{definition}
					{\bf full functor}
					{\bf faithfull functor}
			\end{definition}
	\end{frame}

	
	\begin{frame}{GO THROUGH THIS SLIDE}
			\begin{definition}
					{\bf natural transformation}
			\end{definition}
			\begin{definition}
					{\bf natural isomorphism} or {\bf equivalence}
					{\bf isomorphic} or {\bf equivalent}
			\end{definition}
	\end{frame}



	\begin{frame}{GO THROUGH THIS SLIDE}
			\begin{definition}
					{\bf isomorphic} or {\bf equivalent}
			\end{definition}

			\begin{definition}
					{\bf essential image}
					{\bf isomorphism-dense} or {\bf essentially surjective}
			\end{definition}
	\end{frame}
	
	
	
	\begin{frame}{GO THROUGH THIS SLIDE}
			\begin{definition}
					An {\bf equivalence} between categories $\mathcal{A}$ and $\mathcal{B}$ consists of
					a pair $\mathcal{A}\overset{F}{\underset{G}{\rightleftarrows}}\mathcal{B}$
					of functors together with natural isomorphisms
							$$
									\eta:1_A\rightarrow G\circ F,
							$$
							$$
									 \varepsilon:F\circ G\rightarrow 1_B.
							$$
					If there exists an equivalence between $A$ and $B$, we say that $A$ and $B$ are
					{\bf equivalent}, and write $A\cong B$.
					We also say that the functors $F$ and $G$ are {\bf equivalences}.
			\end{definition}
			\begin{definition}
					Let $\mathcal{A}$ be a category. A {\bf subcategory} $\mathcal{S}$ of $\mathcal{A}$
					consists of a sub $\mathrm{ob}(\mathcal{S})$ of $\mathrm{ob}(\mathcal{A})$ together with,
					for each $S, S' \in\mathrm{ob}(\mathcal{S})$, a subclass $\mathcal{S}(S, S')$ of $\mathcal{A}(S, S')$,
					such that $\mathcal{S}$ is closed under composition and identities.
					It is a {\bf full subcategory} if $\mathcal{S}(S, S') = \mathcal{A}(S, S')$ for all $S, S'\in\mathrm{ob}(\mathcal{S})$.
			\end{definition}
	\end{frame}
	
	
	
	\begin{frame}{GO THROUGH THIS SLIDE}
			\begin{lemma}[Corollary 1.3.19 in "Basic Category Theory", T. Leinster]
					Let $F: \mathcal{C} \rightarrow \mathcal{D}$ be a full and faithful functor. Then $\mathcal{C}$ is
					equivalent to the full subcategory $\mathcal{C'}$
					of $\mathcal{D}$ whose objects are those of the form
					$F(C)$ for some $C\in \mathcal{C}$.
			\end{lemma}
			\begin{proposition}[Prop 1.3.18 in "Basic Category Theory", T. Leinster]
					 {\bf essentially surjective on objects}
			\end{proposition}
			\begin{theorem}
					Let $F:\mathcal{C}\rightarrow\mathcal{D}$ be a functor. Then, the following propositions are equivalent:
							\begin{enumerate}
									\item $F$ is an equivalence.
									\item $F$ is full, faithful and essentially surjective.
									\item $F$ is a part of some adjoint equivalence $(F, G, \eta, \varepsilon)$.
							\end{enumerate}
			\end{theorem}
	\end{frame}
	
	
	
	\begin{frame}{GO THROUGH THIS SLIDE}
			\begin{lemma}
					Let $F:\mathcal{C}\rightarrow\mathcal{D}$ be a functor. Then, the following propositions are equivalent:
							\begin{enumerate}
									\item $F$ is an equivalence.
									\item $F$ is full, faithful and essentially surjective.
							\end{enumerate}
			\end{lemma}
			\begin{proof}
					By the previous lemma, we only have to take an equivalence
					$\mathbf{M}\overset{F}{\rightarrow}\mathrm{Mon}$; i.e. 
							$$
									{}^\exists G:\mathcal{D}\rightarrow\mathcal{C}:functor,
									F\circ G \cong \mathrm{Id}_{\mathcal{C}} \wedge
									G\circ F \cong \mathrm{Id}_{\mathcal{D}}
							$$
			\end{proof}
	\end{frame}
	


	\begin{frame}{GO THROUGH THIS SLIDE}
			\begin{proof}
							$(\rightarrow)$ Given a monoid $M \in\mathrm{Mon}$, then $\mathcal{C}$, a pair of sets,
							is a category when defined as follows:
							$$
									\mathrm{Ob}(\mathbf{M}):=\{*\}
							$$
							$$
									\mathrm{Mor}(\mathbf{M}):=M
							$$
							$$
									\circ_{\mathcal{M}}:=\circ_{\mathrm{Mon}}.
							$$
							$(\leftarrow)$ Let $S$ be a category of $\mathbf{M}$, then $S$ is a monoid when defined as follows:
							$$
									\mathrm{Ob}(M):=\mathrm{Ob}(\mathcal{C})
							$$
							$$
									\mathrm{Mor}(M):=\mathrm{End}_\mathcal{C}(*)=M
							$$
							$$
									\circ_{\mathrm{Mon}}:=\circ_{\mathcal{M}}.
							$$
			\end{proof}
	\end{frame}
	
	

	\begin{frame}
			\begin{definition}[pseudo-orders]
					A relation $\leq$ on a set $P$ is called a {\bf pseudo-order} or a {\bf preorder} if
					it is reflexive and transitive; i.e. for all $a,b,c\in P$, we have that:
							\begin{itemize}
									\item (reflexivity) $a\leq a$
									\item (transitivity) $a\leq b, b\leq c \Rightarrow a\leq c$.
							\end{itemize}
					A set that is equipped with a preorder is called a {\bf preordered set} (or {\bf proset}).
			\end{definition}
			\begin{example}
					\begin{itemize}
							\item partial orders
							\item total orders, or linear orders
							\item equivalence relations
					\end{itemize}
			\end{example}
	\end{frame}



	\begin{frame}{Examples of categories}
			\begin{example}[the category with preordered sets]
					Let $P$ be a poset, and $a,b$ elements in $P$. Then,
					$\mathcal{C}$, a pair of sets, defined by
							\begin{itemize}
									\item $\mathrm{Ob}(\mathcal{C}):=(\mbox{the underlying set of }P)$
									\item $\mathcal{C}(a,b):=\left\{(b,a)\right\}~(\mbox{if }a\leq b)$
									\item $\mathcal{C}(a,b):=\left\{(b,a)\right\}~(otherwise)$
									\item $(c_3,c_2)\circ_{\mathcal{C}}(c_2,c_1):=(c_3,c_1)~
									({}^\forall c_1,{}^\forall c_2, {}^\forall c_3\in P)$
									(, where $\circ_\mathcal{C}$ in $\mathcal{C}$)
							\end{itemize}
					is a category.
			\end{example}
	\end{frame}
	
	
	
	\begin{frame}{Examples of categories}
			\begin{example}[the category with totally ordered sets]
					Let $P$ be an ordered set, and $a,b$ elements in $P$. Then,
					$\mathcal{C}$, a pair of sets, defined by
							\begin{itemize}
									\item $\mathrm{Ob}(\mathcal{C}):=\{\mbox{the underlying set of }P\}$
									\item $\mathcal{C}(a,b):=\left\{(b,a)\right\}~(\mbox{if }a\leq b)$
									\item $\mathcal{C}(a,b):=\left\{(b,a)\right\}~(otherwise)$
									\item $(c_3,c_2)\circ_{\mathcal{C}}(c_2,c_1):=(c_3,c_1)~
									({}^\forall c_1,{}^\forall c_2, {}^\forall c_3\in P)$
									(, where $\circ_\mathcal{C}$ in $\mathcal{C}$)
							\end{itemize}
					is a category.
			\end{example}
	\end{frame}
	
	
	
	\begin{frame}{Examples of categories}
			\begin{example}[special case for the above]
					For $n\geq 0$, a countable finite totally orderd set $S_{n}:=(\{0,1,\cdots, n-1\},\leq_P)$ is a category.
							$$
												\xymatrix{
														1			
														\ar[r]^{}
														&
														2
														\ar[r]^{}
														&
														3
														\ar[r]^{}
														&
														\cdots
												}
							$$
			\end{example}
			$\rightarrow$ We already had nearly the same chain as follows:
							$$
												\xymatrix{
														1			
														&
														\leq
														&
														2
														&
														\leq
														&
														3
														&
														\leq
														&	
														\cdots
												}
							$$
			$\rightarrow$ The most upside is the {\bf same} as descrete categories.
	\end{frame}


	
	\begin{frame}{Examples of categories}
			\begin{definition}[Zermelo-Fraenkel set theory (ZFC)]
				\begin{itemize}
						\item ({\bf Axiom of extensionality})\\
						Two sets are equal (are the same set) if they have the same elements:\\
						$\forall x\forall y[\forall z(z\in x\Leftrightarrow z\in y)\Rightarrow x=y].$
						\item ({\bf Axiom of regularity} (also called the {\bf Axiom of foundation}))\\
						Every non-empty set x contains a member y such that x and y are disjoint sets:\\
						$\forall x\,(x\neq \varnothing \rightarrow \exists y\in x\,(y\cap x=\varnothing ))$\\
						This implies, for example, that no set is an element of itself and that every set has an ordinal rank.
						\item ({\bf Axiom schema of specification}
						(also called the {\bf axiom schema of separation} or {\bf of restricted comprehension}))
				\end{itemize}
			\end{definition}
	\end{frame}



	\begin{frame}{Examples of categories}
			\begin{definition}[Zermelo-Fraenkel set theory (ZFC)]
				\begin{itemize}
						\item ({\bf Axiom of pairing})\\
						\item ({\bf Axiom of union})\\
						\item ({\bf Axiom schema of replacement})\\
				\end{itemize}
			\end{definition}
	\end{frame}



	\begin{frame}{Examples of categories}
			\begin{definition}
					\begin{itemize}
						\item ({\bf Axiom of infinity})\\
						\item ({\bf Axiom of power set})\\
						For any set $x$, there is a set $y$ that contains every subset of $x$:
						$$\forall x\exists y\forall z[z\subseteq x\Rightarrow z\in y].$$
						\item ({\bf Well-ordering theorem})\\
					\end{itemize}
			\end{definition}
	\end{frame}


	
	\begin{frame}{Examples of categories}
			\begin{axiom}[HERE IN USE]
					An universe $\mathcal{U}$ is fixed.
			\end{axiom}
			\begin{definition}
					An element of $\mathcal{U}$ is called a {\bf small set}.
			\end{definition}
			This expression "small" does not refer to how small its cardinality is.
			\begin{proposition}
					$\{\mathcal{U}\}$ is a finite set with some single element, however
					$\{\mathcal{U}\} \notin \mathcal{U}$ holds.
			\end{proposition}
			\begin{proof}
					First we have $\mathcal{U}\in\{\mathcal{U}\}$ by definition. 
					Assuming $\{\mathcal{U}\}\in \mathcal{U}$ holds,
					by using the definition of universes,
					we have $\mathcal{U}\in\mathcal{U}$ in contradiction to the axiom of regularity.
			\end{proof}
	\end{frame}



	\begin{frame}{Examples of categories}
			\begin{example}
					For small sets $a,b\in\mathcal{U}$,
					a category $\mathcal{C}$ defined by
							\begin{itemize}
									\item $\mathrm{Ob}(\mathcal{C}):=\mathcal{U}$
									\item $\mathcal{C}(a,b):=\{\mbox{maps from }a\mbox{ to }b\}$
									\item ordinary composite of maps 
							\end{itemize}
					is called a category of (small) sets, and is denoted by $\underline{\mathrm{Set}}$. 
			\end{example}
			$\rightarrow$ We want to consider a category of {\bf entire} sets, however we have difficulty
			using that category because that is not a set. Therefore, we compose a category of {\bf small} sets,
			which is a really set.
			\begin{definition}
					A structured set with a small underlying set is called a {\bf small structured set}.
			\end{definition}
	\end{frame}
	
	

	\begin{frame}{Examples of categories}
			\begin{example}
			$\underline{\mathrm{Grp}}$ is a category, where its objects are all small groups $\{G_i\}_{i\in I}$ for an index set $I$,
			its arrow is a group homomorphism of $G_S$ for a set $S$,
			and its composition is the operator in $G_S$.
			$\underline{\mathrm{Grp}}$ is called a {\bf category of (small) groups}.
			\end{example}
			\begin{example}
			$\underline{\mathrm{Mon}}$ is a category, where its objects are all small monoids $\{M_i\}_{i\in I}$ for an index set $I$,
			its arrow is a monoid homomorphism of $M_S$ for a set $S$,
			and its composition is the operator in $M_S$.
			$\underline{\mathrm{Mon}}$ is called a {\bf category of (small) monoids}.
			\end{example}
			\begin{example}
			$\underline{\mathrm{Ab}}$ is a category, where its objects are all small Abelian groups $\{A_i\}_{i\in I}$ for an index set $I$,
			its arrow is an Abelian group homomorphism of $A_S$ for a set $S$,
			and its composition is the operator in $A_S$.
			$\underline{\mathrm{Ab}}$ is called a {\bf category of (small) abelian groups}.
			\end{example}
	\end{frame}
	
	
	
	\begin{frame}{Examples of categories}
			\begin{example}
			$\underline{\mathrm{Ring}}$ is a category, where its objects are all small rings $\{R_i\}_{i\in I}$ for an index set $I$,
			its arrow is a ring homomorphism of $R_S$ for a set $S$,
			and its composition is the operators in $R_S$.
			$\underline{\mathrm{Ring}}$ is called a {\bf category of (small) rings}.
			\end{example}
			\begin{example}
			$\underline{\mathrm{CRing}}$ is a category, where its objects are all small commutative $\{R_i\}_{i\in I}$ for an index set $I$,
			its arrow is a ring homomorphism restericted to $R_S$ for a set $S$,
			and its composition is the operators in $R_S$.
			$\underline{\mathrm{CRing}}$ is called a {\bf category of (small) commutative rings}.
			\end{example}
	\end{frame}
	
	
	
	\begin{frame}{Examples of categories}
			\begin{example}
			$\underline{\mathrm{RMod}}$ is a category, where its objects are all small left $R$-modules,
			its arrows are all linear maps.
			$\underline{R\mathrm{-Mod}}$ is called a {\bf category of (small) left $R$-modules}.
			\end{example}
			\begin{example}
			$\underline{\mathrm{ModR}}$ is a category, where its objects are all small right $R$-modules,
			its arrows are all linear maps.
			$\underline{R\mathrm{-Mod}}$ is called a {\bf category of (small) right $R$-modules}.
			\end{example}
			\begin{example}
			$\underline{\mathrm{Ord}}$ is a category, where its objects are all small ordered sets,
			its arrows are all preserving maps,
			and its composition is regular one of maps.
			$\underline{\mathrm{Ord}}$ is called a {\bf category of (small) ordered sets}.
			\end{example}
	\end{frame}
	
	
	\begin{frame}{Examples of categories}
			\begin{example}
			$\underline{\mathrm{Top}}$ is a category, where its objects are all small topological spaces,
			its arrows are all continuous maps,
			and its composition is the usual composition of maps.
			$\underline{\mathrm{Top}}$ is called a {\bf category of (small) topological spaces}.
			\end{example}
			\begin{example}
			$\underline{\mathrm{Toph}}$ is a category, where its objects are all small topological spaces,
			its arrows are all homotopy classes of continuous maps.
			$\underline{\mathrm{Toph}}$ is called a {\bf category of (small) topological spaces}.
			\end{example}
			\begin{example}
			$\underline{\mathrm{Top}_*}$ is a category, where its objects are topological spaces with selected base point,
			its arrows are all base point-preserving maps.
			$\underline{\mathrm{Top}_*}$ is called a {\bf category of (small) topological spaces}.
			\end{example}
	\end{frame}

	
	
	\begin{frame}{Examples of categories}
			\begin{example}
			$C^r\underline{\mathrm{Mfd}}$ is a category, where its objects are all small $C^r$-manifolds,
			its arrows are all $C^r$-maps.
			$\underline{C^r\mathrm{Mfd}}$ is called a {\bf category of (small) $C^r$-manifolds}.
			\end{example}
			\begin{example}
			$\underline{\mathrm{Sch}}$ is a category, where its objects are all small schemes,
			its arrows are all morphisms of schemes,
			and its composition is the usual composition of maps.
			$\underline{\mathrm{Sch}}$ is called a {\bf category of (small) schemes}.
			\end{example}
	\end{frame}
	
	
	
	\begin{frame}{Examples of categories}
			\begin{example}
			$\underline{\mathrm{Matr}_K}$ for a fixed field $K$ is a category, where its objects are all positive integers $m,n,\cdots$,
			and 
			its arrow is a $m\times n$ matrix $A$ (which is regarded as a map $A:m\rightarrow n$),
			and its composition is the usual matrix product.
			$\underline{\mathrm{Matr}_K}$ is called a {\bf category of (small) vector spaces}.
			\end{example}
			\begin{example}
			$\underline{\mathrm{Vct}_K}$ for a fixed field $K$ is a category, where its objects are all small vector spaces over $K$,
			its arrows are all linear transformations,
			and its composition is usual composition of maps.
			$\underline{\mathrm{Vct}_K}$ is called a {\bf category of (small) vector spaces}.
			\end{example}
	\end{frame}



	\begin{frame}{Examples of categories}
			\begin{example}
			$\underline{\mathrm{Euclid}}$ is a category, where its objects are all small Euclidean spaces,
			its arrows are all orthogonal transformations.
			$\underline{\mathrm{Euclid}}$ is called a {\bf category of (small) Euclidean spaces}.
			\end{example}
			\begin{example}
			$\underline{\mathrm{Ses-}A}$ is a category, where its objects are all small short exaxt sequences of $A$-modules.
			$\underline{\mathrm{Ses-}A}$ is called a {\bf category of (small) $A$-modules}.
			\end{example}
	\end{frame}
	
	
	
		\begin{frame}{Examples of categories}
			\begin{example}
			$\underline{\mathrm{Set}_*}$ is a category, where its objects are all small sets each with a selected base-point,
			its arrows are all base-point preserving maps.
			$\underline{\mathrm{Set}_*}$ is called a {\bf category of (small) base points}.
			\end{example}
			\begin{example}
			$\underline{\mathrm{Smgrp}}$ is a category, where its objects are all small semigroups,
			its arrows are all semigroup morphisms.
			$\underline{\mathrm{Smgrp}}$ is called a {\bf category of (small) semigroups}.
			\end{example}
	\end{frame}



	\begin{frame}{Examples of categories}
			\begin{example}
			$\underline{\mathrm{Met}}$ is a category, where its objects are all small metric spaces $X,Y,\cdots$,
			its arrows $X\rightarrow Y$ those functions which preserve the metric,
			and its composition is usual multiplication of real numbers.
			$\underline{\mathrm{Met}}$ is called a {\bf category of (small) metric spaces}.
			\end{example}
	\end{frame}

	
	
	
	\begin{frame}{Examples of categories}
			\begin{definition}
					A category $\mathcal{C}$ is small if it is small as a set; i.e. $\mathcal{O}$ and $\mathcal{M}$ are small.
			\end{definition}
			\begin{example}[small categories]
					$\underline{\mathrm{Set}},\underline{\mathrm{Grp}},\underline{\mathrm{Ab}},\underline{\mathrm{Top}}$
			\end{example}
			\begin{counterexample}[small categories]
					$\underline{\mathrm{Set}},\underline{\mathrm{Grp}},\underline{\mathrm{Ab}},\underline{\mathrm{Top}}$
			\end{counterexample}
	\end{frame}



	\begin{frame}{Examples of categories}
			\begin{definition}
					A category $\mathcal{C}$ is called an $\mathcal{U}-$category if
							$$
									\mathcal{C}(a,b)\in\mathcal{U}
							$$
					holds for objects $a,b\in\mathrm{Ob}(\mathcal{C})$.
			\end{definition}
			\begin{example}[$\mathcal{U}-$categories]
					$\underline{\mathrm{Set}},\underline{\mathrm{Grp}},\underline{\mathrm{Ab}},\underline{\mathrm{Top}}$
			\end{example}
	\end{frame}



\section{Main contents Part1}






\section{Main contents Part2}







\section{Main contents Part3}






\section{Main contents Part4}






\section{Conclusion at 21 May, 2019}
	\begin{frame}{Conclusion at 21 May, 2019}
		\begin{itemize}
 			\item the defintion of categories
					\begin{itemize}
							\item a "directed graph" together with
									\begin{itemize}
											\item associative composite regarding arrows
											\item the identity arrow
									\end{itemize}
					\end{itemize}
			\item the examples of categories
					\begin{itemize}
							\item $\mathrm{Top}$ of topological spaces and homeomorphisms.
							\item $\mathrm{Vect}_K$ of vector spaces over a field $K$ and homomorphisms.
							\item $\mathrm{Mon}$ of monoids and hoomorphisms restricted to them.
							\item more other examples...
					\end{itemize}
			\item we make sure to verify
			$\mathcal{M}\cong\mathrm{Mon}$ ($\mathcal{M}$ is a subcategory with single object,
			of a category)
		\end{itemize}
	\end{frame}



\section{Introduction3}
	\begin{frame}{Definition of functors}
			\begin{definition}[Functors]
					Let 
							\begin{itemize}
									\item $\mathcal{C},\mathcal{D}:\mbox{categories}$.
									\item $F_0: \mathrm{Ob}(\mathcal{C})\rightarrow \mathrm{Ob}(\mathcal{D}):
									\mbox{a map}$.
									\item $F_1: \mathrm{Mor}(\mathcal{C})\rightarrow \mathrm{Mor}(\mathcal{D}):
									\mbox{a map}$,  for $ a,b\in \mathcal{C}$, assignning to the
									 homset $\mathcal{C}(a,b)$ the other one
									 $\mathcal{D}(F_0(a),F_0(b))$ ; i.e.
									${}^\forall x,y\in\mathcal{C}, {}^\forall f\in\mathcal{C}(x,y), s(F_1(f))=F_0(x), t(F_1(f))=F_0(y)$.

							\end{itemize}
					Then,
					the quadruplet $F=(\mathcal{C},\mathcal{D},F_0,F_1)$
					is called a {\bf functor} from $\mathcal{C}$ to $\mathcal{D}$, denoted by $F:\mathcal{C}\rightarrow\mathcal{D}$, if
							\begin{itemize}
									\item $F_1$ preserves composition:
									$^\forall (f,g)\in\mathcal{M}_2(\mathcal{C}), F_1(g\circ f)=F_1(g)\circ F_1(f)$
									\item  $F_1$ preserves identity morphisms:
									$^\forall x\in\mathrm{Ob}(\mathcal{C}), F_1(1_x)=1_{F_0(x)}$
							\end{itemize}
					all satisfy $F$.
					
			\end{definition}
	\end{frame}
	
	
	
	\begin{frame}{Examples of functors}
			\begin{definition}[Contravariant functors]
					Let 
							\begin{itemize}
									\item $\mathcal{C},\mathcal{D}:\mbox{categories}$.
									\item $F_0: \mathrm{Ob}(\mathcal{C})\rightarrow \mathrm{Ob}(\mathcal{D}):
									\mbox{a map}$.
									\item $F_1: \mathrm{Mor}(\mathcal{C})\rightarrow \mathrm{Mor}(\mathcal{D}):
									\mbox{a map}$,  for $ a,b\in \mathcal{C}$, assignning to the
									 homset $\mathcal{C}(a,b)$ the other one
									 $\mathcal{D}(F_0(b),F_0(a))$ ; i.e.
									${}^\forall x,y\in\mathcal{C}, {}^\forall f\in\mathcal{C}(x,y), s(F_1(f))=F_0(y), t(F_1(f))=F_0(x)$.

							\end{itemize}
					Then,
					the quadruplet $F=(\mathcal{C},\mathcal{D},F_0,F_1)$
					is called a {\bf contravariant functor} from $\mathcal{C}$ to $\mathcal{D}$,
					denoted by $F:\mathcal{C}\rightarrow\mathcal{D}$, if
							\begin{itemize}
									\item $F_1$ preserves composition:
									$^\forall (f,g)\in\mathcal{M}_2(\mathcal{C}), F_1(g\circ f)=F_1(f)\circ F_1(g)$
									\item  $F_1$ preserves identity morphisms:
									$^\forall x\in\mathrm{Ob}(\mathcal{C}), F_1(1_x)=1_{F_0(x)}$
							\end{itemize}
					all satisfy $F$.
					
			\end{definition}
	\end{frame}

	
	
	
	\begin{frame}{Examples of functors}
			\begin{example}[the identity functor]
					For a category $\mathcal{C}$, the quadruplet $\mathrm{Id}_\mathcal{C}=(\mathcal{C},\mathcal{C},
					(\mathrm{Id}_\mathcal{C})_0,(\mathrm{Id}_\mathcal{C})_1)$ defined by
							\begin{itemize}
									\item $(\mathrm{Id}_\mathcal{C})_0(x):=x$
									\item $(\mathrm{Id}_\mathcal{C})_1(f):=f$
							\end{itemize}
					is a functor, called the {\bf identity functor}.
			\end{example}
	\end{frame}



	\begin{frame}{Examples of functors}
			\begin{example}[]
					For a category $\mathcal{C}$, the quadruplet $(?)_\sharp=(\underline{\mathrm{Set}},
					\underline{\mathrm{Set}},
					((?)_\sharp)_0,((?)_\sharp)_1)$ defined by
							\begin{itemize}
									\item $((?)_\sharp)_0(X):=\mathcal{P}(X)$
							\end{itemize}
					and, for small sets $X,Y\in\underline{\mathrm{Set}}$ and an arrow $f\in\mathcal{C}(X,Y)$,
							\begin{itemize}
									\item $((?)_\sharp)_1(f):\mathcal{P}(X)\rightarrow\mathcal{P}(Y);S\mapsto f(S)
									$
							\end{itemize}
					is a functor.
			\end{example}
			\begin{example}[]
					For a category $\mathcal{C}$, the quadruplet $(?)^\sharp=(\underline{\mathrm{Set}},
					\underline{\mathrm{Set}},
					((?)^\sharp)_0,((?)^\sharp)_1)$ defined by
							\begin{itemize}
									\item $((?)^\sharp)_0(X):=\mathcal{P}(X)$
							\end{itemize}
					and, for small sets $X,Y\in\underline{\mathrm{Set}}$ and an arrow $g\in\mathcal{D}(Y,X)$,
							\begin{itemize}
									\item $((?)^\sharp)_1(g):\mathcal{P}(Y)\rightarrow\mathcal{P}(X);S\mapsto f^{-1}(S)
									$
							\end{itemize}
					is a contravariant functor.
			\end{example}
	\end{frame}



	\begin{frame}{Examples of functors}
			\begin{example}[]
					For a category $\mathcal{C}$, the quadruplet $\mathcal{O}=(\underline{\mathrm{Top}},
					\underline{\mathrm{Set}},
					\mathcal{O}_0,\mathcal{O}_1)$ defined by
							\begin{itemize}
									\item $\mathcal{O}_0(X):=\{\mbox{the collection of all open sets of }X\}$
							\end{itemize}
					and, for small topological spaces $X,Y\in\underline{\mathrm{Top}}$ and
					a continuous arrow $f\in\mathcal{C}(X,Y)$,
							\begin{itemize}
									\item $\mathcal{O}_1(f):
									\mathcal{O}_0(Y)\rightarrow \mathcal{O}_0(X);U\mapsto f^{-1}(U)
									$
							\end{itemize}
					is a contravariant functor.
			\end{example}
	\end{frame}



	\begin{frame}
			\begin{definition}
					For an ordered set $(P,\leq_P)$, a subset $I$ of $P$ is called a {\bf poset ideal} of $(P,\leq_P)$ if the following holds:
					$${}^\forall a\in P, {}^\forall b\in I, a\leq_P b \Rightarrow a\in I$$
			\end{definition}
			\begin{example}
					The quadruplet
					$\mathcal{I}=(\underline{\mathrm{Ord}},
					\underline{\mathrm{Set}},
					(\mathcal{I})_0,(\mathcal{I})_1)$ defined by
							\begin{itemize}
									\item $(\mathcal{I})_0((X,\leq_X)):=\{\mbox{all poset ideals of }(X,\leq_X)\}$
							\end{itemize}
					and, for small ordered sets $(P,\leq_P),(Q,\leq_Q)\in\underline{\mathrm{Ord}}$ and an arrow
					$f\in\underline{\mathrm{Ord}}((Q,\leq_Q),(P,\leq_P))$,
							\begin{itemize}
									\item $(\mathcal{I})_1(f):\mathcal{I}_0((Q,\leq_Q))\rightarrow\mathcal{I}_0((P,\leq_P));
									J\mapsto f^{-1}(J)
									$
							\end{itemize}
					(, really well-defined, ) is a contravariant functor.
			\end{example}
	\end{frame}



	\begin{frame}
			\begin{lemma}
					The "well-defined" property on the previous example is true.
			\end{lemma}
			\begin{proof}
					For $I_1$, a poset ideal of $(Q,\leq_Q)$,
					$$f^{-1}(I_1)\in(\mathcal{I})_0((P,\leq_P))$$
					$$\Leftarrow f^{-1}(I_1):\mbox{poset ideal of }(P,\leq_P)$$
					$$\Leftarrow{}^\forall a\in P, {}^\forall b\in f^{-1}(I_1), a\leq _P b\Rightarrow a\in f^{-1}(I_1)$$
					$$\Leftarrow {}^\forall a\in P,{}^\exists c\in I_1, f(a)=c.$$
					$$(\Rightarrow {}^\exists a\in P,{}^\forall c\in I_1, f(a)\neq c.)$$
					$$(\Rightarrow <{}^\forall b\in f^{-1}(I_1),
					{}^\exists d\in I_1, f(b)=d> (\because b\in f^{-1}(I_1)):\mbox{contradiction})$$
			\end{proof}
	\end{frame}



	\begin{frame}
			\begin{example}[]
					For a category $\mathcal{C}$, the quadruplet $F=(\underline{\mathrm{Grp}},
					\underline{\mathrm{Set}},
					F_0,F_1)$ defined by
							\begin{itemize}
									\item $F_0((X,\cdot_X)):=X$
							\end{itemize}
					and, for small groups $(G,\cdot_G),(G',\cdot_{G'})\in\underline{\mathrm{Grp}}$,
							\begin{itemize}
									\item $F_1(f):G\rightarrow G' ;(\mbox{an elements-conveyer and a operators-preserver })f
									\mapsto (\mbox{a mere elements-conveyer })f
									$
							\end{itemize}
					is a contravariant functor.
			\end{example}
			\begin{example}
					In the same way, we are going to get the functor to $\underline{\mathrm{Set}}$ from each of
					$$\underline{\mathrm{Ab}},\underline{\mathrm{Rng}},\underline{\mathrm{Top}}.$$		
			\end{example}
	\end{frame}
	
	
	
	\begin{frame}
			\begin{example}
					For
					$$F=(\mathcal{C},\mathcal{D},F_0,F_1), F'=(\mathcal{D},\mathcal{E},F'_0,F'_1):functors,$$
					$$\circ:\mbox{ a composition (of maps in set theory)},$$
					then $F=(\mathcal{C},\mathcal{E},F'_0\circ F_0,F'_1\circ F_1)$ is a functor, called the
					{\bf composite functor }of $G$ with $F$.
			\end{example}
	\end{frame}
	
	
	
	\begin{frame}
			\begin{definition}
					For a category $\mathcal{C}$ and a morphism $\mathcal{C}(A,B)$,
					$f$ is called a {\bf monomorphism} (or $f$ is {\bf mono} for short) if, for
					$C\in\mathcal{C}$, $$f_*:\mathcal{C}(C,A)\rightarrow\mathcal{C}(C,B);g\mapsto f\circ g$$
					is an injection in set theory (i.e. a one-to-one map).
			\end{definition}
	\end{frame}
	
	
	\begin{frame}
			\begin{lemma}
					For an arrow $f\in \underline{\mathrm{Set}}(A,B)$ (i.e. $f$ is a map between small sets),
					the following propositions are equivallent:
							$$f \mbox{ is a mono}.$$
							$$f \mbox{ is an injection in set theory; i.e. a one-to-one map}.$$
							
			\end{lemma}
	\end{frame}



	\begin{frame}
			\begin{lemma}
					For an arrow $f\in \underline{\mathrm{Top}}((X,\mathcal{O}_X),(Y,\mathcal{O}_Y))$,
					 the following equivallence holds:
							$$f \mbox{ is mono}.$$
							$$f \mbox{ is one-to-one}.$$
			\end{lemma}
	\end{frame}



	\begin{frame}
			\begin{definition}
					For a category $\mathcal{C}$ and a morphism $\mathcal{C}(A,B)$,
					$f$ is called an {\bf epimorphism} (or $f$ is {\bf epi} for short) if, for
					$C\in\mathcal{C}$, $$f^*:\mathcal{C}(B,C)\rightarrow\mathcal{C}(A,C);g\mapsto g\circ f$$
					is an injection in set theory (i.e. a one-to-one map).
			\end{definition}
	\end{frame}



	\begin{frame}
			\begin{lemma}
					For an arrow $f\in \underline{\mathrm{Top}}((X,\mathcal{O}_X),(Y,\mathcal{O}_Y))$,
					 the following equivallence holds:
							$$f \mbox{ is mono}.$$
							$$f \mbox{ is one-to-one as a mere map or a "mere-conveyer"}.$$
			\end{lemma}
	\end{frame}
	
	
	
	\begin{frame}
			\begin{definition}
					For a category $\mathcal{C}$ and a morphism $\mathcal{C}(A,B)$,
					$f$ is called an {\bf isomorphism} if the following equaiton holds:
							$${}^\exists g\in\mathcal{C}(B,A),<f\circ_{\mathcal{C}}g=1_B>\wedge<g\circ_{\mathcal{C}}f=1_A>,$$
					where $g$ is called the {\bf inverse} (or the {\bf two-sided inverse}) of $f$.
			\end{definition}
			\begin{proposition}
					On the previous lemma, $g$ is unique.\\
					$(\because \mbox{the inverse of a map is unique in set theory}.)$
			\end{proposition}
			\begin{attention}
					An arrow, which is {\bf mono} and {\bf epi}, between categories is {\bf not} an {\bf isomorphism in general}.
			\end{attention}
	\end{frame}



	\begin{frame}
			\begin{definition}	
					For objects $A,B$ of a category $\mathcal{C}$,
					$A$ and $B$ are isomorphic, denoted by $A\cong B$, if we have an isomorphism from $A$ to $B$;
					i.e. the following equaiton holds:
							$${}^\exists f\in\mathcal{C}(A,B),
							^\exists g_f\in\mathcal{C}(B,A),<f\circ_{\mathcal{C}}g_f=1_B>\wedge<g_f\circ_{\mathcal{C}}f=1_A>.$$
			\end{definition}
	\end{frame}



\section{Conclusion at 27 May, 2019}
	\begin{frame}{Conclusion at 27 May, 2019}
		\begin{itemize}
 			\item the defintion of functors
					\begin{itemize}
							\item an arrow between categories, with
									\begin{itemize}
											\item identity-equivallence
											\item operators-equivallence
									\end{itemize}
					\end{itemize}
			\item the examples of functors
					\begin{itemize}
							\item the identity functor.
							\item $f^\sharp=(\underline{\mathrm{Set}},\underline{\mathrm{Set}},
							(f^\sharp)_0,(f^\sharp)_1)$.
							\item $f_\sharp=(\underline{\mathrm{Set}},\underline{\mathrm{Set}},
							(f_\sharp)_0,(f_\sharp)_1)$.
							\item $\mathcal{O}=(\underline{\mathrm{Top}},\underline{\mathrm{Set}},\mathcal{O}_0,\mathcal{O}_1)$.
							\item forgetful functors.
							\item composite functors.
							\item more other examples (but was raised up)...
					\end{itemize}
			 \item a couple of defintions relevant to categories 
					\begin{itemize}
							\item mono
							\item epi
							\item isomorphisms between categories
							\item isomorphism between objects in a category
					\end{itemize}

		\end{itemize}
	\end{frame}








%\begin{frame}{今後の課題}
%\begin{itemize}
% \item プログラムの数学的背景の研究.
% \begin{itemize}
%\item 圏論の更なる勉強.
%\begin{itemize}
%\item{\rm tripleable theorem}.
%\item$\Omega$代数.
%\end{itemize}
%\item 数理論理学
%\begin{itemize}
%\item 排中律
%\item 直観主義論理
%\end{itemize}
%\item オートマトンの計算モデルの関係計算式による定式化.
%\item 関係データベースのデータモデルの関係計算式による定式化.
% \end{itemize}
%\item 関係計算のCoqライブラリの実装
% \begin{itemize}
%\item Coqによる実装のための基礎の勉強.
%\item 修士論文でまとめた形式化の{\rm Coq}による実装.
%\item 関係代数のその他の性質の{\rm Coq}による実装.
%\item 関係代数の性質の形式証明.
%\item 関係計算により定式化された実問題の性質の形式証明.
% \end{itemize}
%\end{itemize}


%\end{frame}












%%%%%%%%%%%%%%%%%%%%%%%%%%%%%%%%%%%%%%%%%%%%%%%%%%%%%%%%%%%%%%%%%%%%%%%%%
%\subsection{参考文献}
%\begin{frame}[allowframebreaks]{参考文献}
 %\scriptsize
 %\bibliographystyle{plain}
 %\nocite{furusawa2015,maddux1991,demorgan1864,peirce1883}
 %\bibliography{jsacc2015}
%\begin{frame}{参考文献}

%{\scriptsize
%\begin{thebibliography}{99}
%\setlength{\itemsep}{-.5zw}
%\beamertemplatetextbibitems

%\bibitem{barr} Michael Barr, {\sl Relational algebra}, Lecture Notes in Math., 
%137:39-55, 1970.

%\end{thebibliography}
%}

%\end{frame}









\end{document}
